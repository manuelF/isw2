\section{Dise\~no}

  \subsection{Diagrama de clases}

  \paragraph{Abstracci\'on de los sensores}
  El enunciado habla sobre tres tipos de sensores bien concretos.
  Decidimos modelar a cada uno de ellos por serparado. Es decir, no realizamos
  una abstracci\'on para Sensor. El motivo para sustentar esta decisi\'on
  que se refleja tanto a nivel de dise\~no como a nivel implementativo es
  el siguiente.

  Si bien los sensores se comportan de forma semejante,
  o mejor dicho, son polimorficos con respecto al mensaje \texttt{Sensar},
  no lo son vistos desde la perspectiva de un lenguaje est\'aticamente tipado.
  Esto es porque cada sensor devuelve un valor de un tipo o clase distinta.
  Una soluci\'on a esto hubiese sido no modelar los tipos de los valores de
  retorno con distintas clases, y poner \'unicamente un valor num\'erico
  (al estilo \textit{double}). Nos
  pareci\'o que perd\'iamos mucha sem\'antica con esta soluci\'on.

  Otra soluci\'on podr\'ia haber sido utilizar el patr\'on \textit{visitor}.
  Pero nos pareci\'o que complejiza el modelo y la implementaci\'on innecesariamente.

  Al no haber realizado esta abstracci\'on se podr\'ia objetar que restringimos
  la extensibilidad en cuanto a m\'as o distinto tipo de sensores.
  Para responder a esta posible objeci\'on debe observarse el modelado del
  \condiciones, el de los sensores y el del
  \arduino. Al desligar el sensor del \arduino conseguimos
  sensores vers\'atiles, en el sentido de que pueden depender de varios sensores
  reales (es decir del mundo real). M\'as a\'un, los sensores modelados
  permiten tener l\'ogica interna para manejar los sensores del mundo real
  correctamente.
  (FALTA nombrar patron composit)

  Si se quisieran agregar nuevos tipos de sensores deber\'ian agregarse nuevas
  clases de sensores al modelo. Por supuesto deber\'a tambi\'en modificarse
  el c\'odigo de \condiciones para que se comunique
  con los nuevos sensores. Pero consideramos que esto es b\'asicamente inevitable,
  por m\'as que se realice una abstracci\'on del sensor, pues la l\'ogica
  de todo el sistema depender\'a, en este caso, de nuevos par\'ametros.

  \paragraph{Abstracci\'on de los actuadores}
  Para el caso de los actuadores nos encontramos con una situaci\'on semejante
  a la reci\'en presentada. En este caso, sin embargo, optamos por realizar
  una abstracci\'on. Esta se denomina \texttt{Actuador Simple}. El nombre
  refleja la naturaleza sencilla de los actuadores modelados: b\'asicamente
  responden al mensaje \texttt{Suministrar} con una cantidad. Donde
  \texttt{Cantidad} es una clase que representa valores discretos y que adem\'as
  son interpretados por cada actuador de forma independiente.

  Esta abstracci\'on permite realizar una calibraci\'on de cada actuador a la
  hora de inicializar el sistema, que queda
  guardada en el actuador. Por otro lado permite, al \decisiones,
  devolver decisiones en un formato semejante al almacenado en el
  \texttt{Plan Maestro} (y el descripto en el enunciado), que \'unicamente
  especifica cantidades aproximadas, las modeladas en la clase
  \texttt{Cantidades}.

  \paragraph{Interacci\'on entre \decisiones y \condiciones}
  Inicialmente decidimos tener un \timer que peri\'odicamente llame a \condiciones
  con el mensaje \sensarCondiciones. Una vez recopilada la informaci\'on de los
  sensores, \condiciones mandaba el mensaje \tomarDecisiones a \decisiones.

  El problema con este protocolo es que \decisiones depende de \condiciones
  para entrar en juego. Por otro lado, \condiciones termina dependiendo de
  \condiciones a nivel dise\~no e implementaci\'on, lo cual no resulta razonable,
  pues son partes independientes del sistema y este acoplamiento puede ser evitado.

  Para esto ustamo dos \timer. Los objetos que antes estaban acoplados, ahora
  pueden actuar libremente, siendo activados por \timer. \condiciones, luego
  de sensar, escribe los resultados en el historial. \decisiones lee estos
  resultados al ser activado, y toma una decisi\'on.

  Otro aspecto interesante que surgi\'o al analizar esta interacci\'on es
  el comportamiento estilo \textit{observer} que se da entre \timer y
  \condiciones y entre \timer y \decisiones. Intentamos utilizar el patr\'on
  cl\'asico en el dise\~no, pero no result\'o natural. Los motivos son
  principalmente dos:
  \begin{itemize}
      \item El \timer se comporta como un observable, pero tiene una sutileza:
          debe ajustarse el tiempo. Si bien esto puede ser solucionado de
          forma prolija agregando objetos, decidimos que complicaba el dise\~no
          por una cuesti\'on \'unicamente formal, que no facilitaba nada
          concreto.
      \item Siempre que se siga usando al \timer como tal, el dise\~no seguir\'a
          siendo extensible, en este aspecto. Pues la funcionalidad de \timer
          no deber\'ia cambiar, por la esencia misma de un \timer.
  \end{itemize}
  Por estos motivos, creemeos que la extensibilidad no fue restringida al
  no utilizar un \textit{observer} cl\'asico.




  \paragraph{\cliente y \servidor}
  Para que el sistema pueda funcionar real\'isticamente, nos result\'o esencial
  desacoplar totalmente el funcionamiento autom\'atico del mismo: manejo de
  actuadores, recopilar informaci\'on, tomar decisiones, etc. Del funcionamiento
  asincr\'onico debido al uso por parte del usuario: guardar entradas sobre la
  planta en el historial, realizar consultas, etc.

  Para esto separamos el programa en dos procesos. El cliente y el servidor.
  Que a su vez, dieron lugar a dos objetos: \cliente y \servidor.

  \paragraph{Bootstrapping}

  \paragraph{Mensaje}

  \paragraph{Historial}

  \subsection{Diagramas de objetos}

  \subsection{Diagramas de FALTA}
