\section{Dise\~no}

  \subsection{Diagrama de clases}
    El diagrama de clases puede encontrarse en el anexo de diagramas y figuras.
    Notar que se incluye un diagrama id\'entico pero que muestra de forma aproximada
    como se corresponden las clases implementadas con las \textit{stories} del
    \textit{backlog}.

      \subsubsection{Abstracci\'on de los sensores}
          El enunciado habla sobre tres tipos de sensores bien concretos.
          Decidimos modelar a cada uno de ellos por separado. Es decir, no realizamos
          una abstracci\'on para Sensor. El motivo para sustentar esta decisi\'on
          que se refleja tanto a nivel de dise\~no como a nivel implementativo es
          el siguiente.

          Si bien los sensores se comportan de forma semejante,
          o mejor dicho, son polim\'orficos con respecto al mensaje \texttt{Sensar},
          no lo son vistos desde la perspectiva de un lenguaje est\'aticamente tipado.
          Esto es porque cada sensor devuelve un valor de un tipo o clase distinta.
          Una soluci\'on a esto hubiese sido no modelar los tipos de los valores de
          retorno con distintas clases, y poner \'unicamente un valor num\'erico
          (al estilo \textit{double}). Nos
          pareci\'o que perd\'iamos mucha sem\'antica con esta soluci\'on.

          Otra soluci\'on podr\'ia haber sido utilizar el patr\'on \textit{visitor} sobre
          un sensor abstracto, de manera que externamente se utilizara el sensor y se lo
          interpretara como la medici\'on apropiada. Esto nos pareci\'o que complejiza el
          modelo y la implementaci\'on innecesariamente, adem\'as de agregar acoplamiento
          entre funcionalidades claramente delimitadas.

          Al no haber realizado esta abstracci\'on se podr\'ia objetar que restringimos
          la extensibilidad en cuanto a m\'as o distinto tipo de sensores.
          Para responder a esta posible objeci\'on, debe observarse el modelado del
          \condiciones{}, el de los sensores y el del \arduino{}.
          Al desligar el sensor del \arduino{} conseguimos
          sensores vers\'atiles, en el sentido de que pueden depender de uno o varios sensores
          reales (es decir del mundo real). M\'as a\'un, los sensores modelados
          permiten tener l\'ogica interna para manejar los sensores del mundo real
          correctamente.

          Si se quisieran agregar nuevos tipos de sensores deber\'ian agregarse nuevas
          clases de sensores al modelo. Por supuesto deber\'a tambi\'en modificarse
          el c\'odigo de \condiciones{} para que se comunique
          con los nuevos sensores. Pero consideramos que esto es b\'asicamente inevitable,
          por m\'as que se realice una abstracci\'on del sensor, pues la l\'ogica
          de todo el sistema depender\'a, en este caso, de nuevos par\'ametros.

      \subsubsection{Abstracci\'on de los actuadores}
          Para el caso de los actuadores nos encontramos con una situaci\'on semejante
          a la reci\'en presentada. En este caso, sin embargo, optamos por realizar
          una abstracci\'on. Esta se denomina \texttt{Actuador Simple}. El nombre
          refleja la naturaleza sencilla de los actuadores modelados: b\'asicamente
          responden al mensaje \texttt{Suministrar} con una cantidad. Donde
          \texttt{Cantidad} es una clase que representa valores discretos y que adem\'as
          son interpretados por cada actuador de forma independiente.

          Esta abstracci\'on permite realizar una calibraci\'on de cada actuador a la
          hora de inicializar el sistema, que queda
          guardada en el actuador. Por otro lado permite, al \decisiones{},
          devolver decisiones en un formato semejante al almacenado en el
          \texttt{Plan Maestro} (y el descripto en el enunciado), que \'unicamente
          especifica cantidades aproximadas, las modeladas en la clase
          \texttt{Cantidades}.

      \subsubsection{Clase \texttt{Cantidad} y \calibrador{}}
          Justamente para que la cantidad tenga la sem\'antica apropiada en cada
          contexto, es necesario que el actuador sepa interpretarla.
          Para esto se lo debe calibrar al inicializar el sistema, de esto se
          encarga el \calibrador{}. Adem\'as la calibraci\'on podr\'ia ser incluida
          en el plan maestro para soportar distintos tipos de planta de forma
          c\'omoda. Si bien esto no estaba en el enunciado y no fue implementado,
          el modelo resulta extensible para estas modificaciones, ya que resulta
          esperable que se necesite este tipo de funcionalidad en el futuro.

      \subsubsection{Interacci\'on entre \decisiones{} y \condiciones{}}
          Inicialmente decidimos tener un \timer{} que peri\'odicamente llame a \condiciones{}
          con el mensaje \sensarCondiciones{}. Una vez recopilada la informaci\'on de los
          sensores, \condiciones{} mandaba el mensaje \tomarDecisiones{} a \decisiones{}.

          El problema con este protocolo es que \decisiones{} depende de \condiciones{}
          para entrar en juego. Por otro lado, \condiciones{} termina dependiendo de
          \condiciones{} a nivel dise\~no e implementaci\'on, lo cual no resulta razonable,
          pues son partes independientes del sistema y este acoplamiento puede ser evitado.

          Para esto usamos dos \timer{}. Los objetos que antes estaban acoplados, ahora
          pueden actuar libremente, siendo activados por \timer{}. \condiciones{}, luego
          de sensar, escribe los resultados en el \historial{}. \decisiones{} lee estos
          resultados al ser activado, y toma una decisi\'on.

          Otro aspecto interesante que surgi\'o al analizar esta interacci\'on es
          el comportamiento estilo \textit{observer} que se da entre \timer{} y
          \condiciones{} y entre \timer{} y \decisiones{}. Intentamos utilizar el patr\'on
          cl\'asico en el dise\~no, pero no result\'o natural. Los motivos son
          principalmente dos:
          \begin{itemize}
              \item El \timer{} se comporta como un observable, pero tiene una sutileza:
                  debe ajustarse el tiempo. Si bien esto puede ser solucionado de
                  forma prolija agregando objetos, decidimos que complicaba el dise\~no
                  por una cuesti\'on \'unicamente formal, que no facilitaba nada
                  concreto.
              \item Siempre que se siga usando al \timer{} como tal, el dise\~no seguir\'a
                  siendo extensible, en este aspecto. Pues la funcionalidad de \timer{}
                  no deber\'ia cambiar, por la esencia misma de un \timer{}.
          \end{itemize}
          Por estos motivos, creemeos que la extensibilidad no fue restringida al
          no utilizar un \textit{observer} cl\'asico.

      \subsubsection{\cliente{} y \servidor{}}
          Para que el sistema pueda funcionar al momento de implementarlo, nos result\'o esencial
          desacoplar totalmente el funcionamiento autom\'atico del mismo: manejo de
          actuadores, recopilar informaci\'on, tomar decisiones, etc. Del funcionamiento
          asincr\'onico debido al uso por parte del usuario: guardar entradas sobre la
          planta en el \historial{}, realizar consultas, etc.

          Para esto separamos el programa en dos procesos. El cliente y el servidor.
          Que a su vez, dieron lugar a dos objetos: \cliente{} y \servidor{}.

          El patr\'on utilizado es \textit{Fa\c{c}ade}. El \servidor{} es qui\'en
          encapsula todo el comportamiento autom\'atico del sistema.

          La comunicaci\'on con el cliente no es trivial, y se detalla su dise\~no a
          continuaci\'on.

          Para ver como se implement\'o la comunicaci\'on efectivamente, remitirse
          a la secci\'on de implementaci\'on.
          
      \subsubsection{Clase \mensaje{}}
          Al tener que transmitir los mensajes entre \cliente{} y \servidor{} entre
          procesos distintos, result\'o natural y conveniente especificar la forma
          y el prop\'osito de estos mensajes.
          Para esto utilizamos el patr\'on \textit{Command}. Esto da extensibilidad
          a la hora de agregar nuevas funcionalidades en \servidor{}, pues se pueden
          agregar nuevos comandos/mensajes comodamente.

          Como los procesos corriendo en un sistema operativo \textsc{UNIX} \'unicamente
          pueden comunicarse enviando \textit{bytes}, es decir, el sistema operativo no
          provee niveles de abstracci\'on para enviar objetos, la clase
          \constructorMensaje{} viene al caso. Y la abstracci\'on del \mensaje{}
          resulta muy \'util.
          El funcionamiento b\'asico es: para comunicarse el \cliente{} con el \servidor{}
          (y viceversa), el objeto crea un \mensaje{}.
          Luego, utilizando el m\'etodo \texttt{Serializar} de \mensaje{}, consigue un
          \textit{string} que representa al mensaje, y que env\'ia mediante un \textit{socket}.

          El objeto que recibe esto reconstruye el mensaje utilizando un \constructorMensaje{}
          y luego se env\'ia este mensaje.

          Si bien el proceso puede parecer complejo, es el \textit{tradeoff} m\'as simple
          \footnote{Que se nos ocurri\'o.} para que la soluci\'on no se salga del paradigma,
          pero al mismo tiempo, poder separar el cliente y el servidor en procesos distintos.

          Como se dijo arriba, remitirse a la secci\'on de implementaci\'on para ver
          los detalles de la comunicaci\'on, incluyendo c\'omo se ejecuta el mensaje
          una vez reconstruido por el servidor, mediante \textit{double dispatch}.

      \subsubsection{Clase \historial{}}
          El \historial{} tiene actualmente tres tipos de entradas, pero esto puede
          ser f\'acilmente modificado, gracias a utilizar herencia.
          La parte interesante del \historial{} es la forma en que es recorrido.
          Si bien esta funcionalidad no fue implementada si fue dise\~nada.
          Para recorrer el \historial{} se utiliza un \textit{Visitor}, en este caso
          representado por \recopilador{} quien sabe como recorrer \historial{}.

      \subsubsection{Dise\~no del \planmaestro{}}


  \subsection{Diagramas de objetos}
    En la secci\'on de diagramas se pueden observar dos diagramas de objetos.
    Uno detalla el estado del sistema haciendo omisi\'on de la parte que concierne
    al \historial{}.
    El segundo muestra el estado del \historial{} y algunos objetos pertinentes
    en un intervalo de unos pocos minutos de tiempo. Esto se debe a que el
    \historial{} almacena una gran cantidad de informaci\'on en poco tiempo.

  \subsection{Diagramas de secuencia}
    Los diagramas de secuencia muestran las dos acciones principales que
    representan el ciclo de vida del sistema.
    Una es una iteraci\'on del \decisiones{}, la otra, una iteraci\'on de \condiciones{}.
