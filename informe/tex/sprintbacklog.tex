\def \CODI {INT1}
\def \USRI {\aag  check the PH, humidity and temperature of
    the ground through the Ardruino sensor \sic verify the actual plant conditions.}

\def \ACCI {
    \empezarenum
        \item The values displayed must be the ones measured by the sensors at
            the moment they're required.
    \terminarenum
    }

\def \VALI {7}

\def \DESI {Interface between the three sensors and the application.
    Interpret the information, save it (to make decisions) and display it so
    the user can read it.
    }

\def \TASI {
    \empezarenum
        \item simulate the information measured by the Arduino.
        \item retrieve the information measured from the Arduino interface.
        \item interpret the information correctly.
        \item display the information in a human-readable way.
    \terminarenum
}

\def \POII{8}




\def \CODII {INT2}
\def \USRII {\aag  get weather for tomorrow from the metheorological center \sic
    see what actions will be taken.}
\def \ACCII {
    \empezarenum
        \item The values displayed must be the ones forecasted by the central at the moment they're required.
    \terminarenum}

\def \VALII {5}

\def \DESII {
    Interface between the meteorological attachment and the application.
    Interpret the information, store it (to make future decisions), and
    display it so the user can read it.}

\def \TASII {
    \empezarenum
        \item simulate the information from the center.
        \item retrieve the information from the meteorological center.
        \item interpret the information correctly.
        \item display the information in a human-readable way.
    \terminarenum}

\def \POIII {5}



\def \CODIII {INT10}
\def \USRIII {\aag automatize the actions to take
    \sic follow the master plan}

\def \ACCIII {
    \empezarenum
        \item The system must automatically check the sensors value.
        \item The system must check with the master plan to check whether any
            actions have to be taken.
    \terminarenum
    }

\def \VALIII {7}

\def \DESIII {
    The application must have enough logic to decide the next action to follow
    based on the information retreived from the sensors.
}

\def \TASIII {
    \empezarenum
        \item (INT1).
        \item (GAL2).
        \item decide what actions to take.
    \terminarenum
}

\def \POIIII {8}

    

\def \CODIV {INT12}
\def \USRIV {
    \aag automatize the actuators activation
    \sic so actions can be taken automatically
}

\def \ACCIV {
    \empezarenum
        \item The system must trigger the actuators if any actions have to be taken.
    \terminarenum
}
\def \VALIV {10}

\def \DESIV {
    The decitions made int (INT10) must be followed, by activating the actuators acordingly.
}

\def \TASIV {
    \empezarenum
        \item simulate the actuators incidence and response.
        \item (INT10).
        \item control the actuators.
    \terminarenum}

\def \POIIV {5}



\def \CODV {GAL2}
\def \USRV {\aab input a master growth plan \sic specify growth conditions for the plant}

\def \ACCV {
    \empezarenum
        \item The system must have a valid way to input the values for every growth stage.
    \terminarenum}
\def \VALV {5}

\def \DESV {Interface that allows the botanist especify values for the PH, humidity and
    temperature of the ground for every growth stage.}

\def \TASV {
    \empezarenum
        \item desing an interface so the user can input a master plan.
    \terminarenum}

\def \POIV {3}





\begin{landscape}
\section{\sprintback}

\begin{small}
\begin{tabular}{ |l|\ancho|\ancho|l|l|\ancho|\ancho| }
\hline
\multicolumn{6}{ |c| }{Sprint Backlog} \\
\hline
Code & User Story & Acceptance Criteria & Value & Points & Tasks & Description \\
\hline
\CODI & \USRI & \ACCI & \VALI & \POII & \TASI & \DESI \\
\hline
\CODII & \USRII & \ACCII & \VALII & \POIII & \TASII & \DESII \\
\hline
\CODIII & \USRIII & \ACCIII & \VALIII & \POIIII & \TASIII & \DESIII \\
\hline
\CODIV & \USRIV & \ACCIV & \VALIV & \POIIV & \TASIV & \DESIV \\
\hline
\CODV & \USRV & \ACCV & \VALV & \POIV & \TASV & \DESV \\
\hline
\end{tabular}

\end{small}
\end{landscape}
