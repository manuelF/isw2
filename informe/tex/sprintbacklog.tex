\def \aag {\textbf{As a Gardener I Want to }}
\def \sic {\textbf{So I Can }}

\def \USRI {\aag  check the PH, humidity and temperature of
    the ground through the Ardruino sensor \sic verify the actual plant conditions.}

\def \ACCI {
    \empezarenum
        \item The values displayed must be the ones measured by the sensors at
            the moment they're required.
    \terminarenum
    }

\def \VALI {7}

\def \DESI {Interface between the three sensors and the application.
    Interpret the information, save it (to make decisions) and display it so
    the user can read it.
    }

\def \TASI {
    \empezarenum
        \item simulate the information measured by the Arduino.
        \item retrieve the information measured from the Arduino interface.
        \item interpret the information correctly.
        \item display the information in a human-readable way.
    \terminarenum
}

\def \POII{8}

\def \USRII {\aag  get weather for tomorrow from the metheorological center \sic
    see what actions will be taken.}
\def \ACCII {
    \empezarenum
        \item The values displayed must be the ones forecasted by the central at the moment they're required.
    \terminarenum}

\def \VALII {5}

\def \DESII {
    Interface between the meteorological attachment and the application.
    Interpret the information, store it (to make future decisions), and
    display it so the user can read it.}

\def \TASII {
    \empezarenum
        \item simulate the information from the center.
        \item retrieve the information from the meteorological center.
        \item interpret the information correctly.
        \item display the information in a human-readable way.
    \terminarenum}

\def \POIII {5}

\def \USRIII {\aag automatize the actions to take
    \sic follow the master plan}

\def \ACCIII {
    \empezarenum
        \item The system must automatically check the sensors value.
        \item The system must check with the master plan to check whether any
            actions have to be taken.
    \terminarenum
    }

\def \VALIII {7}

\def \DESIII {
    The application must have enough logic to decide the next action to follow
    based on the information retreived from the sensors.
}

\def \TASIII {
    \empezarenum
        \item (INT1).
        \item (GAL2).
        \item decide what actions to take.
    \terminarenum
}

\def \POIIII {8}

    
\def \USRIV {
    \aag automatize the actuators activation
    \sic so actions can be taken automatically
}

\def \ACCIV {
    \empezarenum
        \item The system must trigger the actuators if any actions have to be taken.
    \terminarenum
}
\def \VALIV {10}

\def \DESIV {
    The decitions made int (INT10) must be followed, by activating the actuators acordingly.
}

\def \TASIV {
    \empezarenum
        \item simulate the actuators incidence and response.
        \item (INT10).
        \item control the actuators.
    \terminarenum}

\def \POIIV {5}

%Total for theme 'Interface'
%62.0 points
%Garden Logic
%Code:
%GAL
%
%    GAL1
%    As
%    botanist
%    
%    specify care rules
%    \sic
%    get warnings when the growth conditions are not ideal
%        [edit]
%    [edit]
%    3
%    GAL2
%    In progress
%    sprint 1
%    As
%    botanist
%    
%    input a master growth plan
%    \sic
%    specify growth conditions for the plant
%        a)
%        The system must have a valid way to input the values for every growth stage.
%    Value: 5
%
%    Description:
%    Interface that allows the botanist especify values for the PH, humidity and temperature of the ground for every growth stage.
%
%    Tasks:
%    -desing an interface so the user can input a master plan.
%    3
%    GAL3
%    As
%    a botanist
%    
%    visualize the historical values of indicators and supplies.
%    \sic
%    decide if the values are correct.
%        [edit]
%    [edit]
%    3
%    Add story
%
%Total for theme 'Garden Logic'
%9.0 points
%Web
%Code:
%WEB
%
%    WEB1
%    As
%    a gardener
%    
%    check via web the status
%    \sic
%    monitor my plant from around the world
%
%

\def \ancho {p{5.4cm}}


\begin{landscape}
\section{Sprint Backlog}

\begin{small}
\begin{tabular}{ |\ancho|\ancho|l|l|\ancho|\ancho| }
\hline
\multicolumn{6}{ |c| }{Sprint Backlog} \\
\hline
User Story & Acceptance Criteria & Value & Points & Tasks & Description \\
\hline
\USRI & \ACCI & \VALI & \POII & \TASI & \DESI \\
\hline
\USRII & \ACCII & \VALII & \POIII & \TASII & \DESII \\
\hline
\USRIII & \ACCIII & \VALIII & \POIIII & \TASIII & \DESIII \\
\hline
\USRIV & \ACCIV & \VALIV & \POIIV & \TASIV & \DESIV \\
\hline
\end{tabular}

\end{small}
\end{landscape}
