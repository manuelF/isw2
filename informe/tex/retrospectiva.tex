\section{Retrospectiva}

  \subsection{Asunciones}
    \subsubsection{Fenolog\'ia}
      Por falta de conocimiento en el \'area asumimos que los estad\'ios de maduraci\'on
      de la planta estaban determinados por lapsos fijos de tiempo. Esto quiere decir que
      el paso de una etapa a la otra est\'a solo controlado por el paso de los meses. 
      Esto provoca que el ingreso de indicadores fenol\'ogicos por parte del usuario
      sea meramente informativo.  

  \subsection{Inconvenientes encontrados}
    \subsubsection{Tareas programadas en el \sprintback{}}
      Al momento de definir las tareas no ten\'iamos experiencia en separar y definir
      tareas concretas para un proyecto. A pesar de nuestro intento por modularizar
      las tareas y no olvidar partes esenciales result\'o que b\'asicamente obviamos
      las tareas de modelado y dise\~no. Por otro lado, las tareas de implementaci\'on
      resultaron modularizadas de forma poco conveniente, pues no se correspond\'ian
      con las clases, que fueron dise\~nadas luego. Si bi\'en el \sprintback{} tuvo
      este inconveniente, a la hora de ponernos a trabajar notamos r\'apidamente
      el problema, y logramos distribuirnos las tareas de forma eficiente.
    \subsubsection{Horas estimadas de implementaci\'on}
      Luego de haber ajustado las tareas del \sprintback{} logramos una buena
      estimaci\'on de las horas de dise\~no, pues habiamos experimentado cuanto
      nos pod\'ia tomar. Por otro lado las horas de implementaci\'on quedaron un tanto
      ajustadas, si bien la diferencia con las horas reales de implementaci\'on no
      fue tan grande.

    \subsection{Pr\'oxima iteraci\'on}

      \subsubsection{Decisiones reales}

      \subsubsection{Historial visitor}

