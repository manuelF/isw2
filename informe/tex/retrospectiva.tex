\section{Retrospectiva}

  \subsection{Asunciones}
    \subsubsection{Fenolog\'ia}
      Por falta de conocimiento en el \'area asumimos que los estad\'ios de maduraci\'on
      de la planta estaban determinados por lapsos fijos de tiempo. Esto quiere decir que
      el paso de una etapa a la otra est\'a solo controlado por el paso de los meses. 
      Esto provoca que el ingreso de indicadores fenol\'ogicos por parte del usuario
      sea meramente informativo.  

  \subsection{Inconvenientes encontrados}
    \subsubsection{Tareas programadas en el \sprintback{}}
      Al momento de definir las tareas no ten\'iamos experiencia en separar y definir
      tareas concretas para un proyecto. A pesar de nuestro intento por modularizar
      las tareas y no olvidar partes esenciales result\'o que b\'asicamente obviamos
      las tareas de modelado y dise\~no. Por otro lado, las tareas de implementaci\'on
      resultaron modularizadas de forma poco conveniente, pues no se correspond\'ian
      con las clases, que fueron dise\~nadas luego. Si bi\'en el \sprintback{} tuvo
      este inconveniente, a la hora de ponernos a trabajar notamos r\'apidamente
      el problema, y logramos distribuirnos las tareas de forma eficiente.
    \subsubsection{Horas estimadas de implementaci\'on}
      Luego de haber ajustado las tareas del \sprintback{} logramos una
      estimaci\'on aceptable de las horas de dise\~no, pues habiamos experimentado cuanto
      nos pod\'ia tomar. Por otro lado las horas de implementaci\'on quedaron un tanto
      ajustadas. La diferencia total fue de $19$ horas.

    \subsection{Pr\'oxima iteraci\'on -- No implementado}
      \subsubsection{Decisiones dependientes del \textit{input} del usuario}
        En lo que se implement\'o las decisiones de como actuar dependen del plan
        maestro, y de los datos recolectados por los sensores.
        Las caracter\'isticas escritas por el usuario sirven a modo de \textit{log}
        para que el usuario pueda chequear la evoluci\'on de la planta.

        En una nueva iteraci\'on se podr\'ia modificar el proceso de decisi\'on para
        que tome en cuenta las caracter\'isticas escritas por el usuario en el
        historial.

        Para esto deber\'ia agregarse un nuevo tipo de entrada en el historial
        que permita guardar caracter\'isticas bien definidas con un formato apropiado.

        Para agregar esto al dise\~no bastar\'a heredar un nuevo tipo de entrada
        del historial, y que \decisiones{} use esta nueva clase.

      \subsubsection{SMS}
        El dise\~no contempla la opci\'on de enviar un \textsc{SMS} en caso de que las
        condiciones cambien abruptamente, esto se har\'ia mediante un objeto encargado
        de leer las entradas de condiciones externas, analizarlas y obrar en caso de
        ser necesario. Esto fue modelado como \texttt{Encargado avisos urgentes}.

      \subsubsection{Implementaci\'on del \calibrador{}}
        En una de las \'ultimas reuniones con el \textit{product owner} se nos hizo notar que
        no estaba contamplada la posibilidad de graduar las \textit{Cantidades} que los 
        actuadores deben interpretar para poder cuidar de la planta. Por ello se 
        modific\'o el dise\~no creando un \calibrador{}, aunque el mismo no se 
        icluy\'o en el \textit{sprint}. La implementaci\'on del mismo no deber\'ia ser
        problem\'atica, pues deben agregarse mensajes en \servidor{} e implementar
        la clase \calibrador{}.
        
      \subsubsection{Implementaci\'on del \recopilador{}}
        Recordemos que el historial posee tres tipos de entrada distintos, es por
        esto que en el siguiente \textit{sprint} se debe implementar el \textit{Visitor} \recopilador{}
        El mismo se encargar\'ia de navegar el historial y recuperar las entradas
        requeridas evitando romper el encapsulamiento de la implementaci\'on del \historial{}.
