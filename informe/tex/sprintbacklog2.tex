\def \diseno {\textbf{Design}}
\def \implementacion {\textbf{Implementation}}
\def \testing {\textbf{Testing}}

\def \codI {INT1}
\def \usrI {\aag{}  check the PH, humidity and temperature of
    the ground through the Ardruino sensor \sic{} verify the actual plant conditions.}

\def \tasI {
    \diseno{}
    \begin{enumerate}
        \item abstract the \textit{Arduinos}.
        \item design the sensors.
        \item design the sensor manager.
        \item design a class or classes for a types that represent
            the measurments.
        \item design an atomatic way for the sensors to sensate and save the
            information.
    \end{enumerate}
    \implementacion{}
    \begin{enumerate}
        \item retrieve the information measured from the Arduino interface.
        \item interpret the information correctly.
        \item display the information in a human-readable way.
    \end{enumerate}
    \testing{}
    \begin{enumerate}
        \item simulate the information measured by the Arduino.
    \end{enumerate}
}

\def \essI {
    \empezarenum
        \item \diseno{} $4 hs$
        \item \implementacion{} $20 hs$
        \item \testing{} $1 hs$
    \terminarenum
}


\def \asstoI {
    \diseno{}
    \begin{enumerate}
        \item $1$ hora  - Luis
        \item $1$ hora  - Luis
        \item $1$ hora  - Luis
        \item $1$ hora  - Fabricio
        \item $2$ horas - Guillermo
	\end{enumerate}
    \implementacion{}
    \begin{enumerate}
        \item $7$ horas - Manuel
        \item $7$ horas - Manuel
        \item $4$ horas - Luciano y Guillermo
    \end{enumerate}
    \testing{}
    \begin{enumerate}
        \item $5$ horas - Luciano y Manuel
    \end{enumerate}
}


\def \codII {INT2}
\def \usrII {\aag{}  get weather for tomorrow from the metheorological center \sic{}
    see what actions will be taken.}

\def \tasII {
    \diseno{}
    \begin{enumerate}
        \item design the metheorological center.
        \item design a class for a type that represents the measurements.
        \item design an atomatic way for the metheorological center to sensate
            and save the information.
    \end{enumerate}
    \implementacion{}
    \begin{enumerate}
        \item retrieve the information from the meteorological center.
        \item interpret the information correctly.
        \item display the information in a human-readable way.
    \end{enumerate}

    \testing
    \begin{enumerate}
        \item simulate the information from the center.
    \end{enumerate}
}

\def \essII {
    \empezarenum
        \item \diseno{} $2hs$
        \item \implementacion{} $4hs$
        \item \testing{} $4hs$
    \terminarenum
}

\def \asstoII {
    \diseno{}
    \begin{enumerate}
        \item $1$ hora - Guillermo y Luis
        \item $1$ hora - Fabricio
        \item $1$ hora - Guillermo
    \end{enumerate}
    \implementacion{}
    \begin{enumerate}
        \item $2$ horas - Manuel
        \item $1$ horas - Manuel
        \item $2$ horas - Luciano
    \end{enumerate}
    \testing{}
    \begin{enumerate}
        \item $5$ horas - Luciano
    \end{enumerate}
}


\def \codIII {INT10}
\def \usrIII {\aag{} automatize the actions to take
    \sic{} follow the master plan}
\def \tasIII {
    \diseno{}
    \begin{enumerate}
        \item design a decision-maker.
        \item design a class for the type of the decisions.
    \end{enumerate}
    \implementacion{}
    \begin{enumerate}
        \item (INT1).
        \item (GAL2).
        \item decide what actions to take.
    \end{enumerate}
}

\def \essIII {
    \empezarenum
        \item \diseno{} $4hs$
        \item \implementacion{} $20hs$
        \item \testing{} $4hs$
    \terminarenum
}

\def \asstoIII {
    \diseno{}
    \begin{enumerate}
        \item $6$ horas - Luis y Guillermo
        \item $1$ hora  - Luis y Guillermo
    \end{enumerate}
    \implementacion{}
    \begin{enumerate}
        \item $20$ horas - Manuel y Luciano
    \end{enumerate}
    \testing{}
    \begin{enumerate}
        \item $4$ horas - Manuel
    \end{enumerate}

}



\def \codIV {INT12}
\def \usrIV {
    \aag{} automatize the actuators activation
    \sic{} so actions can be taken automatically
}

\def \tasIV {
    \diseno{}
    \begin{enumerate}
        \item design the actuators.
        \item design an actuator manager.
    \end{enumerate}
    \implementacion{}
    \begin{enumerate}
        \item (INT10).
        \item control the actuators.
    \end{enumerate}
    \begin{enumerate}
        \item simulate the actuators incidence and response.
    \end{enumerate}
}

\def \essIV {
    \empezarenum
        \item \diseno{} $2hs$
        \item \implementacion{} $4hs$
        \item \testing{} $4hs$
    \terminarenum
}

\def \asstoIV {
    \diseno{}
    \begin{enumerate}
        \item $1$ hora - Guillermo y Luis
        \item $1$ hora - Guillermo y Fabricio
    \end{enumerate}
    \implementacion{}
    \begin{enumerate}
        \item $5$ horas - Luciano
    \end{enumerate}
    \testing{}
    \begin{enumerate}
        \item $6$ horas - Manuel
    \end{enumerate}
}


\def \codV {GAL2}
\def \usrV {\aab input a master growth plan \sic{} specify growth conditions for the plant}

\def \tasV {
    \diseno{}
    \begin{enumerate}
        \item design an interface so the user can input a master plan.
        \item design the master plan format.
    \end{enumerate}
    \implementacion{}
    \begin{enumerate}
        \item implement the master plan in a persistent and yet modifiable
            fashion.
    \end{enumerate}
}

\def \essV {
    \empezarenum
        \item \diseno{} $2hs$
        \item \implementacion{} $4hs$
        \item \testing{} $4hs$
    \terminarenum
}

\def \asstoV {
    \diseno{}
    \begin{enumerate}
        \item $4$ horas - Guillermo y Luis
        \item $1$ hora - Fabricio
    \end{enumerate}
    \implementacion{}
    \begin{enumerate}
        \item $8$ horas - Manuel y Luciano
    \end{enumerate}
    \testing{}
    \begin{enumerate}
        \item $2$ horas - Manuel y Luciano
    \end{enumerate}
}




\def \anchosprintdos {p{10cm}}

\begin{landscape}

Notar que cuando aparecen $n$ horas asignadas a m\'as de
una persona, quiere decir que estas personas estuvieron $n$
horas \textbf{en total} realizando esta tarea; no que que
cada una estuvo $n$ horas.

\begin{small}
\begin{tabular}{ |l|p{4cm}|p{7.5cm}|p{5cm}|p{5cm}| }
\hline
\multicolumn{5}{ |c| }{Sprint Backlog w/ elaborated tasks} \\
\hline
Code & User Story & Tasks & Estimated hours & Assigned to and used hours\\
\hline
\codI & \usrI & \tasI & \essI & \asstoI \\
\hline
\codII & \usrII & \tasII & \essII & \asstoII \\
\hline
\end{tabular}

\begin{tabular}{ |l|p{4cm}|p{7.5cm}|p{5cm}|p{5cm}| }
\hline
\multicolumn{5}{ |c| }{Sprint Backlog w/ elaborated tasks (cont.)} \\
\hline
Code & User Story & Tasks & Estimated hours & Assigned to and used hours \\
\hline
\codIII & \usrIII & \tasIII & \essIII & \asstoIII \\
\hline
\codIV & \usrIV & \tasIV & \essIV & \asstoIV \\
\hline
\codV & \usrV & \tasV & \essV & \asstoV \\
\hline
\end{tabular}

\end{small}
\end{landscape}
