\section{Implementaci\'on}
    \subsection{Testing}
      \paragraph{Para la integridad del \textit{software}}

        Para corroborar el correcto funcionamiento del sistema se crearon objetos para
        representar a los sensores y que simulararn a las mediciones.
        De forma an\'aloga se crearon objetos que simulen a los actuadores.
        
        Se utiliz\'o una \textit{suite} de tests para verificar que las
        serializarciones (en el caso de la comuncicaci\'on v\'ia \textit{sockets})
        y los mensajes y construcciones entre objetos fuesen compatibles.

      \paragraph{Para la \demostracion{}}
        Se probaron tanto instanciaciones de \arduino{} que devolvieran mediciones
        fijas como instanciaciones que devuelvan valores aleatorios o prefijados
        en archivos.
    \subsection{\textit{Script} de la \demostracion{}}

      En lo siguiente se pone a prueba el ciclo de vida b\'asico del sistema:
      despertar a los sensores y actuadores peri\'odicamente, tomar decisiones
      y guardar todo esto en el historial.

      Adem\'as se muestra c\'omo el usuario puede modificar el plan maestro, y,
      lo que es m\'as interesante, como el sistema actua conforme a este cambio.

      \begin{enumerate}
        \item Se corre el servidor (\texttt{./server}).
        \item El servidor carga el plan maestro.
        \item Se abre el cliente (\texttt{./client localhost}).
        \item Se le pide el estado fenol\'ogico a la planta y se
            agregan dos anotaciones.
            La respuesta es que la planta se encuentra en la etapa $4$, donde
            la humedad debe ser baja, el PH alto y la temperatura baja.
        \item El \timer{} de \condiciones{} se activa varias veces. As\'i se guardan las
            mediciones de los sensores y de la central meteorol\'ogica.
            Se observan los valores: PH bajo, Humerdad alto, Temperatura baja y
            Probabilidad de lluvia $10\%$.
            Todo esto se guarda en el historial.
        \item Luego de algunos sensados se activa el \timer{} de \decisiones{}\footnote{
            los tiempos son $1m$ para el de \condiciones{} y $5m$ para el de \decisiones{}.}.
        \item \decisiones{} consulta el \'ultimo sensado en el historial y pregunta al plan
            maestro la estapa actual.
        \item Dados estos valores decide una cantidad para cada actuador: baja, nada, baja.
        \item Env\'ia la decisi\'on a \manejadorAct{}.
        \item \manejadorAct{} se comunica con los actuadores indicando la cantidad
            correspondiente.
        \item Cada actuador se comunica con su arduino indicando el nuevo valor de su
            \textit{dimmer}.
        \item El tomador de decisiones guarda la decisi\'on tomada en el historial.
        \item Se realiza una nueva ronda de sensados.
        \item El usuario pide al cliente modificar el plan maestro.
        \item Se modifica el estado $4$ del plan. Ahora la humedad debe ser baja,
            \textbf{el PH bajo} y la temperatura alta.
        \item Se muestra en pantalla, del servidor, que se modific\'o el plan.
        \item Luego de otro sensado se activa nuevamente el \timer{} de \decisiones{}.
        \item Se prosigue de forma an\'aloga pero administrando una cantidad
            \textbf{distinta} pues el plan es distinto.
      \end{enumerate}

    \subsection{Comunicaci\'on por \textit{Sockets}}
      La comunicaci\'on entre cliente y servidor se realiza utilizando \textsc{UNIX}
      \textit{sockets}.
      Cuando se crea el objeto servidor, se configura un \textit{file descriptor} con
      un \textit{socket} \textit{bindeado} a la primera IP no-local que se encuentre.
      El servidor se pone en modo escucha, esperando una conexi\'on \textsc{TCP}
      al puerto configurado. Al finalizar la conexi\'on el servidor se vuelve a poner
      en modo escucha.
      Es importante notar que por m\'as que el servidor est\'e esperando mensajes,
      simultaneamente, funcionan todos los eventos asicr\'onicos de timers, con lo cual,
      ambas funcionalidades est\'an desacopladas.

      Por la contraparte, la atenci\'on de la comunicaci\'on funciona de la siguiente manera.
      El servidor espera mensajes a traves del \textit{file descriptor} mencionado antes.
      Cuando recibe una cadena de texto (que es lo unico que puede enviarse a traves de
      \textit{sockets}) se la env\'ia a \constructorMensaje{} que se encarga de
      parsear y reconstruir el mensaje original.
      Una vez reconstruido, el servidor ejecuta el mensaje, pasandose a si mismo como par\'ametro.
      Mediante este \textit{double dispatch} el mensaje sabe que debe ejecutar cierta funcionaldidad
      y devolver un mensaje de respuesta. El servidor entonces env\'ia este mensaje serializado
      de vuelta al cliente. El mensaje de respuesta se maneja con un mecanismo an\'alogo.


%      \subsubsection{Bootstrapping}
%      FALTA
